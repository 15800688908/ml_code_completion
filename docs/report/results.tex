%\vspace{-5pt}
\section{Experiments and Results}
\label{sec:results}

In this section, we evaluate the methods outlined in Section~\ref{sec:methods},
on Linux kernel and Twisted library. We first present the accuracy of
predictions on the test set, and then present some interesting predictions that
come out from recurring patterns. We implemented the matrix vector model in
Python, and the feed-forward and recurrent models in Keras~\cite{ref:keras}.
Our code is available on github~\cite{ref:codecompletion}.

\subsection{Accuracy}
\label{sec:accuracy}

\begin{table}[h]
  \centering
  \small {
  \begin{tabular}{l l l r r r r r}
    \hline
    Method & Known & Abs & Top 3 & Key & Pos \\
    Win, \#Keys & acc. & acc. & acc.  & acc. & acc. \\
    \hline
    NL-3, 40, 2000 & 64.4 & 53.4 & 81.7 & 78.0 & 43.7\\
    NL-4, 64, 2000 & 63.2 & 50.6 & 82.0 & 72.8 & 41.8\\
    Attn, 40, 1000 & 67.5 & 56.0 & 83.6 & 79.9 & 48.4\\
    Random & x & x & x & x & x\\
    \hline
  \end{tabular}
  }
  \caption{Test accuracy (\%) of predictions for Linux project}
  \label{tab:linux}
\end{table}

\begin{table}[h]
  \centering
  \small {
  \begin{tabular}{l l l r r r r r}
    \hline
    Method & Known & Abs & Top 3 & Key & Pos \\
    Win, \#Keys & acc. & acc. & acc.  & acc. & acc. \\
    \hline
    NL-3, 40, 500 & 46.3 & x & 64.8 & 64.0 & 14.0\\
    NL-4, 40, 500 & 43.0 & 38.7 & 59.0 & 65.5 & 9.2\\
    Attn, 40, 500 & 46.6 & x & 64.7 & 62.5 & 18.5\\
    GRU, 40, 500 & 41.6 & x & 59.1 & 64.4 & 4.9\\
    Random & x & x & x & x & x\\
    \hline
  \end{tabular}
  }
  \caption{Test accuracy (\%) of predictions for Twisted project}
  \label{tab:twisted}
\end{table}

\noindent
Table~\ref{tab:linux} shows results for Linux source, a C project, and
Table~\ref{tab:twisted} for Twisted, a Python networking library. As mentioned
in Section~\ref{sec:dataset}, we use half the files for training and half for
testing.
Column 1
lists the learning method, window size and number of key words.
NL-3 is a feed forward model with 3 non-linearity layers, and NL-4 is a feed
forward model with 4 non-linearity layers. Attn. computes a weighing
parameter for each input token in the window, and uses 3 non-linearity layers.
GRU is a recurrent model.
Additionally, we report numbers for a naive, random
predictor, that predicts a random token as the next token.

Known acc. is the
accuracy for predictions for the cases where the next token is a key token, or
a positional token from the window being considered. Abs acc. is the accuracy
for predictions for all cases, including cases where the next token is
neither a key word nor a seen positional token. These are often new function
names or variable names, or variables outside of our window. Top 3 is the
percentage of cases where the next token is within top 3 predictions, ignoring
the unknown cases (unseen variable and function names).
We report this number, because generally, we are interested in the top few
suggestions as opposed to the top suggestion with a code recommendation system.
Key acc. is the
accuracy with which we predict key tokens correctly, given the next token is a
key token,
and pos. acc. is the accuracy with which we predict positional tokens
correctly, given the next token is a positional token.
These give us insight into how well our methods predict tokens that are not key
tokens.

We found that attention gives the best results among all the methods we tried.
This makes sense because it is also the most general of all models --- it
directly connects all inputs to the output, (unlike GRU where computed states
are chained), and it has an additional attention parameter, that is just a
constant in the simple feed-forward case. We also found that the feed forward
model with 4 non-linearity layers tends to overfit the training data, giving a slightly
lower accuracy on test data. Increasing the word vector dimensions also tends
to result in overfitting. Though regularization can help with overfitting, we
found that reducing vector dimensions and number of layers was also sufficient
to reduce high variance from overfitting.

The accuracy of predicting non-key words, that is, positional tokens is quite
high for Linux kernel, a C project, indicating there is a lot of redundancy/
recurring patterns. The positional token accuracy is lower for Python.
We think this is because Python is a dynamically typed, high level
language with less redundancy, and more terse syntax.
However, in both cases, the top 3 prediction
accuracy is still much higher than a random predictor, indicating that the
model does extract some patterns out of the code.

A plot of the word vectors reveals a significant clustering of similar
words --- {\texttt {uint8\_t}} and {\texttt {uint16\_t}}, different locking and
unlocking calls, and integers. We have omitted these plots from the report for
space. Unlike natural language texts, however,
we did not observe any significant vector relationships between different
different tokens.
